\documentclass{ximera}

\usepackage{epsfig}

\graphicspath{
  {./}
  {figures/}
}

\usepackage{epstopdf}
%\usepackage{ulem}
\usepackage[normalem]{ulem}

\epstopdfsetup{outdir=./}

\usepackage{morewrites}
\makeatletter
\newcommand\subfile[1]{%
\renewcommand{\input}[1]{}%
\begingroup\skip@preamble\otherinput{#1}\endgroup\par\vspace{\topsep}
\let\input\otherinput}
\makeatother

\newcommand{\EXER}{}
\newcommand{\includeexercises}{\EXER\directlua{dofile(kpse.find_file("exercises","lua"))}}

\newenvironment{computerExercise}{\begin{exercise}}{\end{exercise}}

%\newcounter{ccounter}
%\setcounter{ccounter}{1}
%\newcommand{\Chapter}[1]{\setcounter{chapter}{\arabic{ccounter}}\chapter{#1}\addtocounter{ccounter}{1}}

%\newcommand{\section}[1]{\section{#1}\setcounter{thm}{0}\setcounter{equation}{0}}

%\renewcommand{\theequation}{\arabic{chapter}.\arabic{section}.\arabic{equation}}
%\renewcommand{\thefigure}{\arabic{chapter}.\arabic{figure}}
%\renewcommand{\thetable}{\arabic{chapter}.\arabic{table}}

%\newcommand{\Sec}[2]{\section{#1}\markright{\arabic{ccounter}.\arabic{section}.#2}\setcounter{equation}{0}\setcounter{thm}{0}\setcounter{figure}{0}}
  
\newcommand{\Sec}[2]{\section{#1}}

\setcounter{secnumdepth}{2}
%\setcounter{secnumdepth}{1} 

%\newcounter{THM}
%\renewcommand{\theTHM}{\arabic{chapter}.\arabic{section}}

\newcommand{\trademark}{{R\!\!\!\!\!\bigcirc}}
%\newtheorem{exercise}{}

\newcommand{\dfield}{{\sf dfield9}}
\newcommand{\pplane}{{\sf pplane10}}
\newcommand{\PPLANE}{{\sf PPLANE10}}

% BADBAD: \newcommand{\Bbb}{\bf}. % Package amsfonts Warning: Obsolete command \Bbb; \mathbb should be used instead.

\newcommand{\R}{\mbox{$\mathbb{R}$}}
\let\C\relax
\newcommand{\C}{\mbox{$\mathbb{C}$}}
\newcommand{\Z}{\mbox{$\mathbb{Z}$}}
\newcommand{\N}{\mbox{$\mathbb{N}$}}
\newcommand{\D}{\mbox{{\bf D}}}
\usepackage{amssymb}
%\newcommand{\qed}{\hfill\mbox{\raggedright$\square$} \vspace{1ex}}
%\newcommand{\proof}{\noindent {\bf Proof:} \hspace{0.1in}}

\newcommand{\setmin}{\;\mbox{--}\;}
\newcommand{\Matlab}{{M\small{AT\-LAB}} }
\newcommand{\Matlabp}{{M\small{AT\-LAB}}}
\newcommand{\computer}{\Matlab Instructions}
\renewcommand{\computer}{M\small{ATLAB} Instructions}
\newcommand{\half}{\mbox{$\frac{1}{2}$}}
\newcommand{\compose}{\raisebox{.15ex}{\mbox{{\scriptsize$\circ$}}}}
\newcommand{\AND}{\quad\mbox{and}\quad}
\newcommand{\vect}[2]{\left(\begin{array}{c} #1_1 \\ \vdots \\
 #1_{#2}\end{array}\right)}
\newcommand{\mattwo}[4]{\left(\begin{array}{rr} #1 & #2\\ #3
&#4\end{array}\right)}
\newcommand{\mattwoc}[4]{\left(\begin{array}{cc} #1 & #2\\ #3
&#4\end{array}\right)}
\newcommand{\vectwo}[2]{\left(\begin{array}{r} #1 \\ #2\end{array}\right)}
\newcommand{\vectwoc}[2]{\left(\begin{array}{c} #1 \\ #2\end{array}\right)}

\newcommand{\ignore}[1]{}


\newcommand{\inv}{^{-1}}
\newcommand{\CC}{{\cal C}}
\newcommand{\CCone}{\CC^1}
\newcommand{\Span}{{\rm span}}
\newcommand{\rank}{{\rm rank}}
\newcommand{\trace}{{\rm tr}}
\newcommand{\RE}{{\rm Re}}
\newcommand{\IM}{{\rm Im}}
\newcommand{\nulls}{{\rm null\;space}}

\newcommand{\dps}{\displaystyle}
\newcommand{\arraystart}{\renewcommand{\arraystretch}{1.8}}
\newcommand{\arrayfinish}{\renewcommand{\arraystretch}{1.2}}
\newcommand{\Start}[1]{\vspace{0.08in}\noindent {\bf Section~\ref{#1}}}
\newcommand{\exer}[1]{\noindent {\bf \ref{#1}}}
\newcommand{\ans}{\textbf{Answer:} }
\newcommand{\matthree}[9]{\left(\begin{array}{rrr} #1 & #2 & #3 \\ #4 & #5 & #6
\\ #7 & #8 & #9\end{array}\right)}
\newcommand{\cvectwo}[2]{\left(\begin{array}{c} #1 \\ #2\end{array}\right)}
\newcommand{\cmatthree}[9]{\left(\begin{array}{ccc} #1 & #2 & #3 \\ #4 & #5 &
#6 \\ #7 & #8 & #9\end{array}\right)}
\newcommand{\vecthree}[3]{\left(\begin{array}{r} #1 \\ #2 \\
#3\end{array}\right)}
\newcommand{\cvecthree}[3]{\left(\begin{array}{c} #1 \\ #2 \\
#3\end{array}\right)}
\newcommand{\cmattwo}[4]{\left(\begin{array}{cc} #1 & #2\\ #3
&#4\end{array}\right)}

\newcommand{\Matrix}[1]{\ensuremath{\left(\begin{array}{rrrrrrrrrrrrrrrrrr} #1 \end{array}\right)}}

\newcommand{\Matrixc}[1]{\ensuremath{\left(\begin{array}{cccccccccccc} #1 \end{array}\right)}}



\renewcommand{\labelenumi}{\theenumi}
\newenvironment{enumeratea}%
{\begingroup
 \renewcommand{\theenumi}{\alph{enumi}}
 \renewcommand{\labelenumi}{(\theenumi)}
 \begin{enumerate}}
 {\end{enumerate}
 \endgroup}

\newcounter{help}
\renewcommand{\thehelp}{\thesection.\arabic{equation}}

%\newenvironment{equation*}%
%{\renewcommand\endequation{\eqno (\theequation)* $$}%
%   \begin{equation}}%
%   {\end{equation}\renewcommand\endequation{\eqno \@eqnnum
%$$\global\@ignoretrue}}

%\input{psfig.tex}

\author{Martin Golubitsky and Michael Dellnitz}

%\newenvironment{matlabEquation}%
%{\renewcommand\endequation{\eqno (\theequation*) $$}%
%   \begin{equation}}%
%   {\end{equation}\renewcommand\endequation{\eqno \@eqnnum
% $$\global\@ignoretrue}}

\newcommand{\soln}{\textbf{Solution:} }
\newcommand{\exercap}[1]{\centerline{Figure~\ref{#1}}}
\newcommand{\exercaptwo}[1]{\centerline{Figure~\ref{#1}a\hspace{2.1in}
Figure~\ref{#1}b}}
\newcommand{\exercapthree}[1]{\centerline{Figure~\ref{#1}a\hspace{1.2in}
Figure~\ref{#1}b\hspace{1.2in}Figure~\ref{#1}c}}
\newcommand{\para}{\hspace{0.4in}}

\usepackage{ifluatex}
\ifluatex
\ifcsname displaysolutions\endcsname%
\else
\renewenvironment{solution}{\suppress}{\endsuppress}
\fi
\else
\renewenvironment{solution}{}{}
\fi

\ifcsname answer\endcsname
\renewcommand{\answer}{}
\fi

%\ifxake
%\newenvironment{matlabEquation}{\begin{equation}}{\end{equation}}
%\else
\newenvironment{matlabEquation}%
{\let\oldtheequation\theequation\renewcommand{\theequation}{\oldtheequation*}\begin{equation}}%
  {\end{equation}\let\theequation\oldtheequation}
%\fi

\makeatother

\newcommand{\RED}[1]{{\color{red}{#1}}} 


\title{Sinks, Saddles, and Sources}

\begin{document}
\begin{abstract}
\end{abstract}
\maketitle

 \label{S:6.7}

The qualitative theory of autonomous differential equations begins with
the observation that many important properties of solutions to constant
coefficient systems of differential equations
\begin{equation} \label{e:C2}
\frac{dX}{dt}=CX
\end{equation}
are unchanged by similarity.  

We call the origin of the linear system \eqref{e:C2} a {\em sink}\index{sink} (or 
{\em asymptotically stable}) if all solutions $X(t)$ satisfy
\[
\lim_{t\to\infty}X(t) = 0.
\]
The origin is a {\em source} if all nonzero solutions $X(t)$ satisfy
\[
\lim_{t\to\infty}||X(t)|| = \infty.
\]
Finally, the origin is a {\em saddle} if some solutions limit to $0$ and some 
solutions grow infinitely large.  Recall also from Lemma~\ref{L:simsoln} that
if $B=P\inv CP$, then $P\inv X(t)$ is a solution to $\dot{X}=BX$ whenever
$X(t)$ is a solution to \eqref{e:C2}.  Since $P\inv$ is a matrix of constants
that do not depend on $t$, it follows that
\[
\lim_{t\to\infty}X(t) = 0 \Longleftrightarrow \lim_{t\to\infty}P\inv X(t) = 0.
\]
or 
\[
\lim_{t\to\infty}||X(t)|| = \infty \Longleftrightarrow \lim_{t\to\infty}||P\inv X(t)|| = \infty.
\]
It follows the origin is $C$ is a {\em sink}\index{sink} (or {\em saddle}\index{saddle} 
or {\em source})\index{source} for \eqref{e:C2} if and only if $P\inv X(t)$ is a 
sink (or saddle or source) for $\dot{X}=BX$.  

\begin{theorem}  \label{C:asympstlin} Consider the system \eqref{e:C2} where 
$C$ is a $2\times 2$ matrix.  
\begin{enumeratea}
\item If the eigenvalues of $C$ have negative real part, then the origin is a sink. 
\item If the eigenvalues of $C$ have positive real part, then the origin is a source.
\item If one eigenvalue of $C$ is positive and one is negative, then the origin is a saddle.
\end{enumeratea}  
\end{theorem}

\begin{proof}  
Lemma~\ref{L:simdettr} states that the similar matrices $B$ and $C$ have the 
same eigenvalues.  Moreover, as noted the origin is a sink, saddle, or source for 
$B$ if and only if it is a sink, saddle, or source for $C$.  Thus, we need only verify 
the theorem for normal form matrices as given in Table~\ref{T:3sys}. 

\noindent (a) \quad If the eigenvalues $\lambda_1$ and $\lambda_2$ are real
and there are two independent eigenvectors, then Chapter~\ref{Chap:Planar},
Theorem~\ref{T:putinform} states that the matrix $C$ is similar to the
diagonal matrix
\[
B = \mattwoc{\lambda_1}{0}{0}{\lambda_2}.
\]
The general solution to the differential equation $\dot{X}=BX$ is
\[
x_1(t) = \alpha_1e^{\lambda_1 t} \AND x_2(t) = \alpha_2e^{\lambda_2 t}.
\]
Since
\[
\lim_{t\to\infty}e^{\lambda_1 t} = 0  = \lim_{t\to\infty}e^{\lambda_2 t},
\]
when $\lambda_1$ and $\lambda_2$ are negative, it follows that
\[
\lim_{t\to\infty} X(t) = 0
\]
for all solutions $X(t)$, and the origin is a sink.  Note that if both of 
the eigenvalues are positive, then $X(t)$ will undergo exponential 
growth and the origin is a source.

\noindent (b) \quad If the eigenvalues of $C$ are the complex conjugates
$\sigma\pm i\tau$ where $\tau\neq 0$, then Chapter~\ref{Chap:Planar},
Theorem~\ref{T:putinform} states that after a similarity transformation
\eqref{e:C2} has the form
\[
\dot{X} = \mattwo{\sigma}{-\tau}{\tau}{\sigma}X,
\]
and solutions for this equation have the form \eqref{e:exp0ev} of
Chapter~\ref{Chap:Planar}, that is,
\[
X(t) = e^{\sigma t}
\mattwo{\cos(\tau t)}{-\sin(\tau t)}{\sin(\tau t)}{\cos(\tau t)}X_0
= e^{\sigma t}R_{\tau t}X_0,
\]
where $R_{\tau t}$ is a rotation matrix\index{rotation!matrix}
(recall \eqref{e:rotmat} of
Chapter~\ref{chap:matrices}).  It follows that as time evolves
the vector $X_0$ is rotated about the origin and then expanded or contracted
by the factor $e^{\sigma t}$.  So when $\sigma<0$, $\lim_{t\to\infty} X(t)=0$
for all solutions $X(t)$.  Hence the origin is a sink and when $\sigma > 0$ 
solutions spiral away from the origin and the origin is a source.

\noindent (c) \quad If the eigenvalues are both equal to $\lambda_1$
and if there is only one independent eigenvector, then
Chapter~\ref{Chap:Planar}, Theorem~\ref{T:putinform} states that after a
similarity transformation \eqref{e:C2} has the form
\[
\dot{X} = \mattwo{\lambda_1}{1}{0}{\lambda_1}X,
\]
whose solutions are
\[
X(t) = e^{t\lambda}\mattwoc{1}{t}{0}{1} X_0
\]
using Table~\ref{T:3sys}(c). Note that the functions
$e^{\lambda_1 t} \AND te^{\lambda_1 t}$ both have limits equal to zero as
$t\to\infty$.  In the second case, use l'H\^{o}spital's rule and the
assumption that $-\lambda_1>0$ to compute
\[
\lim_{t\to\infty} \frac{t}{e^{-\lambda_1 t}} =
  -\lim_{t\to\infty} \frac{1}{\lambda_1 e^{-\lambda_1 t}} = 0.
\]
Hence $\lim_{t\to\infty} X(t) = 0$ for all solutions $X(t)$ and the origin
is asymptotically stable.  Note that initially $||X(t)||$ can grow since
$t$ is increasing.  But eventually exponential decay wins out and solutions
limit on the origin.   Note that solutions grow exponentially when
$\lambda_1 > 0$.  
\end{proof}

Theorem~\ref{C:asympstlin} shows that the qualitative features of the 
origin for \eqref{e:C2} depend only on the eigenvalues of $C$ and not
on the formulae for solutions to \eqref{e:C2}.  This is a much simpler 
calculation.  However, Theorem~\ref{T:det_trace} simplifies the calculation 
substantially further.

\begin{theorem} \label{T:det_trace}
\begin{enumeratea}
\item If $det(C) < 0$, then $0$ is a saddle.
\item If $\det(C) > 0$ and $\trace(C) < 0$, then $0$ is a sink.
\item If $\det(C) > 0$ and $\trace(C) > 0$, then $0$ is a source.
\end{enumeratea}
\end{theorem}

\begin{proof}
Recall from \eqref{e:deteigen} that $\det(C)$ is the product 
of the eigenvalues of $C$. Hence, if $\det(C) < 0$, then the signs of the 
eigenvalues must be opposite, and we have a saddle.  Next, suppose 
$\det(C) > 0$.  If the eigenvalues are real, then the eigenvalues are either 
both positive (a source) or both negative (a sink).  Recall from \eqref{e:treigen} 
that $\trace(C)$ is sum of the eigenvalues and the sign of the trace 
determines the sign of the eigenvalues. Finally, assume the eigenvalues 
are complex conjugates $\sigma\pm i\tau$.  Then $\det(C) = \sigma^2+\tau^2 > 0$
and $\trace(C) = 2\sigma$.  Thus, the sign of the real parts of the complex eigenvalues 
is given by the sign of $\trace(C)$. 
\end{proof}

\subsubsection*{Time Series}

It is instructive to note how the time series $x_1(t)$ damps down to the
origin in the three cases listed in Theorem~\ref{C:asympstlin}.
In Figure~\ref{F:oscil} we present the time series for the three
coefficient matrices:
\begin{align*}
C_1 &= \mattwo{-2}{0}{0}{-1}, \\
C_2 &= \mattwo{-1}{-55}{55}{-1}, \\
C_3 &= \mattwo{-2}{1}{0}{-2}.
\end{align*}
In this figure, we can see the exponential decay to zero associated with the
unequal real eigenvalues of $C_1$; the damped oscillation associated with the
complex eigenvalues of $C_2$; and the initial growth of the time series due
to the $te^{-2t}$ term followed by exponential decay to zero in the equal
eigenvalue $C_3$ example.

\begin{figure*}[htb]
           \centerline{%
           \psfig{file=../figures/expdamp.eps,width=2.2in}
	   \psfig{file=../figures/oscil.eps,width=2.2in}
	   \psfig{file=../figures/grdecay.eps,width=2.2in}}
           \caption{Time series for different sinks.}
           \label{F:oscil}
\end{figure*}


\subsubsection*{Sources Versus Sinks}

The explicit form of solutions to planar linear systems shows that solutions
with initial conditions near the origin grow exponentially in forward time
when the origin of \eqref{e:C2} is a source.  We can prove this point
geometrically, as follows.

The phase planes of sources and sinks are almost the same; they have the
same trajectories but the arrows are reversed.  To verify this point, note
that
\begin{equation}  \label{e:C3}
\dot{X}=-CX
\end{equation}
is a sink when \eqref{e:C2} is a source; observe that the trajectories of
solutions of \eqref{e:C2} are the same as those of \eqref{e:C3} --- just with
time running backwards.  For let $X(t)$ be a solution to \eqref{e:C2}; then
$X(-t)$ is a solution to \eqref{e:C3}.   See Figure~\ref{F:SS} for plots of
$\dot{X}=BX$ and $\dot{X}=-BX$ where
\begin{equation}  \label{E:SS}
B = \mattwo{-1}{-5}{5}{-1}.
\end{equation}

So when we draw schematic phase portraits\index{phase!portrait}
for sinks\index{phase!portrait!for a sink}, we automatically know
how to draw schematic phase portraits for
sources\index{phase!portrait!for a source}.  The trajectories are
the same --- but the arrows point in the opposite direction.

\begin{figure*}[htb]
           \centerline{%
	   \psfig{file=../figures/asink.eps,width=3.2in}
           \psfig{file=../figures/asource.eps,width=3.2in}}
           \caption{(Left) Sink $\dot{X}=BX$ where $B$ is given in
\protect{\eqref{E:SS}}.  (Right) Source $\dot{X}=-BX$.}
           \label{F:SS}
\end{figure*}


\subsection*{Phase Portraits for Saddles}
\index{phase!portrait!for a saddle}

Next we discuss the phase portraits of linear saddles.  Using
{\pplane}\index{\computer!pplane8}, draw the phase portrait
of the saddle
\begin{equation}  \label{e:saddlet}
\begin{array}{rcl}
\dot{x} & = & 2x+y\\
\dot{y} & = & -x-3y,
\end{array}
\end{equation}
as in Figure~\ref{F:linsaddle}.  The important feature of saddles
is that there are special trajectories (the eigendirections) that
limit on the origin in either forward or backward time.

\begin{figure*}[htb]
           \centerline{%
	   \psfig{file=../figures/linpnb.eps,width=3.5in}
           \psfig{file=../figures/linpnts.eps,width=3.5in}}
           \caption{(Left) Saddle phase portrait.
	(Right) First quadrant solution time series.}
           \label{F:linsaddle}
\end{figure*}

\begin{definition} \label{D:stablemfld}
The {\em stable manifold\/} or {\em stable orbit\/} of a saddle consists of
those trajectories that limit on the origin in forward time; the
{\em unstable manifold\/} or {\em unstable orbit\/} of a saddle consists of
those trajectories that limit on the origin in backward time.
\end{definition}
\index{stable!manifold} \index{unstable!manifold}
\index{stable!orbit} \index{unstable!orbit}

Let $\lambda_1<0$ and $\lambda_2>0$ be the eigenvalues of a saddle with
associated eigenvectors $v_1$ and $v_2$.  The stable orbits are given by the
solutions $X(t) = \pm e^{\lambda_1 t}v_1$ and the unstable orbits are given
by the solutions $X(t) = \pm e^{\lambda_2 t}v_2$.

\subsubsection*{Stable and Unstable Orbits using {\sf \pplane}}

The program {\pplane} is programmed to draw the stable and unstable
orbits of a saddle on command. Although the principal use of this
feature is seen when analyzing nonlinear systems, it is useful to
introduce this feature now.  As an example, load the linear system
\eqref{e:saddlet} into {\pplane} and click on {\sf Update}.  Now
pull down the {\sf Analysis} menu and click on {\sf Find nearby
equilibrium}.  Click the cross hairs in the {\sf \PPLANE\; Display}
window on a point near the origin; {\pplane} responds by
plotting the equilibrium and the real eigenvectors --- and by putting 
a small circle about the origin.  The circle indicates that the numerical algorithm
programmed into {\pplane} has detected an equilibrium near
the chosen point.  This process numerically verifies that the origin
is a saddle (a fact that could have been verified in a more
straightforward way).

Now pull down the {\sf Analysis} menu again and click on
{\sf Solve for stable separatrices}.  {\sf \pplane} responds by 
drawing the stable and unstable orbits.
The result is shown in Figure~\ref{F:linsaddle}(left).
On this figure we have also plotted one trajectory
from each quadrant; thus obtaining the phase portrait of a saddle.
On the right of Figure~\ref{F:linsaddle} we have plotted a
time series of the first quadrant solution.  Note how the $x$
time series increases exponentially to $+\infty$ in forward time and 
the $y$ time series decreases in forward time while going exponentially 
towards $-\infty$.  The two time series together
give the trajectory $(x(t),y(t))$ that in forward time is asymptotic
to the line given by the unstable eigendirection.





\includeexercises



\end{document}
