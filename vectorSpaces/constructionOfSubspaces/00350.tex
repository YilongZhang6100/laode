\documentclass{ximera}

\author{Marty Golubitsky}

\usepackage{epsfig}

\graphicspath{
  {./}
  {figures/}
}

\usepackage{epstopdf}
%\usepackage{ulem}
\usepackage[normalem]{ulem}

\epstopdfsetup{outdir=./}

\usepackage{morewrites}
\makeatletter
\newcommand\subfile[1]{%
\renewcommand{\input}[1]{}%
\begingroup\skip@preamble\otherinput{#1}\endgroup\par\vspace{\topsep}
\let\input\otherinput}
\makeatother

\newcommand{\EXER}{}
\newcommand{\includeexercises}{\EXER\directlua{dofile(kpse.find_file("exercises","lua"))}}

\newenvironment{computerExercise}{\begin{exercise}}{\end{exercise}}

%\newcounter{ccounter}
%\setcounter{ccounter}{1}
%\newcommand{\Chapter}[1]{\setcounter{chapter}{\arabic{ccounter}}\chapter{#1}\addtocounter{ccounter}{1}}

%\newcommand{\section}[1]{\section{#1}\setcounter{thm}{0}\setcounter{equation}{0}}

%\renewcommand{\theequation}{\arabic{chapter}.\arabic{section}.\arabic{equation}}
%\renewcommand{\thefigure}{\arabic{chapter}.\arabic{figure}}
%\renewcommand{\thetable}{\arabic{chapter}.\arabic{table}}

%\newcommand{\Sec}[2]{\section{#1}\markright{\arabic{ccounter}.\arabic{section}.#2}\setcounter{equation}{0}\setcounter{thm}{0}\setcounter{figure}{0}}
  
\newcommand{\Sec}[2]{\section{#1}}

\setcounter{secnumdepth}{2}
%\setcounter{secnumdepth}{1} 

%\newcounter{THM}
%\renewcommand{\theTHM}{\arabic{chapter}.\arabic{section}}

\newcommand{\trademark}{{R\!\!\!\!\!\bigcirc}}
%\newtheorem{exercise}{}

\newcommand{\dfield}{{\sf dfield9}}
\newcommand{\pplane}{{\sf pplane10}}
\newcommand{\PPLANE}{{\sf PPLANE10}}

% BADBAD: \newcommand{\Bbb}{\bf}. % Package amsfonts Warning: Obsolete command \Bbb; \mathbb should be used instead.

\newcommand{\R}{\mbox{$\mathbb{R}$}}
\let\C\relax
\newcommand{\C}{\mbox{$\mathbb{C}$}}
\newcommand{\Z}{\mbox{$\mathbb{Z}$}}
\newcommand{\N}{\mbox{$\mathbb{N}$}}
\newcommand{\D}{\mbox{{\bf D}}}
\usepackage{amssymb}
%\newcommand{\qed}{\hfill\mbox{\raggedright$\square$} \vspace{1ex}}
%\newcommand{\proof}{\noindent {\bf Proof:} \hspace{0.1in}}

\newcommand{\setmin}{\;\mbox{--}\;}
\newcommand{\Matlab}{{M\small{AT\-LAB}} }
\newcommand{\Matlabp}{{M\small{AT\-LAB}}}
\newcommand{\computer}{\Matlab Instructions}
\renewcommand{\computer}{M\small{ATLAB} Instructions}
\newcommand{\half}{\mbox{$\frac{1}{2}$}}
\newcommand{\compose}{\raisebox{.15ex}{\mbox{{\scriptsize$\circ$}}}}
\newcommand{\AND}{\quad\mbox{and}\quad}
\newcommand{\vect}[2]{\left(\begin{array}{c} #1_1 \\ \vdots \\
 #1_{#2}\end{array}\right)}
\newcommand{\mattwo}[4]{\left(\begin{array}{rr} #1 & #2\\ #3
&#4\end{array}\right)}
\newcommand{\mattwoc}[4]{\left(\begin{array}{cc} #1 & #2\\ #3
&#4\end{array}\right)}
\newcommand{\vectwo}[2]{\left(\begin{array}{r} #1 \\ #2\end{array}\right)}
\newcommand{\vectwoc}[2]{\left(\begin{array}{c} #1 \\ #2\end{array}\right)}

\newcommand{\ignore}[1]{}


\newcommand{\inv}{^{-1}}
\newcommand{\CC}{{\cal C}}
\newcommand{\CCone}{\CC^1}
\newcommand{\Span}{{\rm span}}
\newcommand{\rank}{{\rm rank}}
\newcommand{\trace}{{\rm tr}}
\newcommand{\RE}{{\rm Re}}
\newcommand{\IM}{{\rm Im}}
\newcommand{\nulls}{{\rm null\;space}}

\newcommand{\dps}{\displaystyle}
\newcommand{\arraystart}{\renewcommand{\arraystretch}{1.8}}
\newcommand{\arrayfinish}{\renewcommand{\arraystretch}{1.2}}
\newcommand{\Start}[1]{\vspace{0.08in}\noindent {\bf Section~\ref{#1}}}
\newcommand{\exer}[1]{\noindent {\bf \ref{#1}}}
\newcommand{\ans}{\textbf{Answer:} }
\newcommand{\matthree}[9]{\left(\begin{array}{rrr} #1 & #2 & #3 \\ #4 & #5 & #6
\\ #7 & #8 & #9\end{array}\right)}
\newcommand{\cvectwo}[2]{\left(\begin{array}{c} #1 \\ #2\end{array}\right)}
\newcommand{\cmatthree}[9]{\left(\begin{array}{ccc} #1 & #2 & #3 \\ #4 & #5 &
#6 \\ #7 & #8 & #9\end{array}\right)}
\newcommand{\vecthree}[3]{\left(\begin{array}{r} #1 \\ #2 \\
#3\end{array}\right)}
\newcommand{\cvecthree}[3]{\left(\begin{array}{c} #1 \\ #2 \\
#3\end{array}\right)}
\newcommand{\cmattwo}[4]{\left(\begin{array}{cc} #1 & #2\\ #3
&#4\end{array}\right)}

\newcommand{\Matrix}[1]{\ensuremath{\left(\begin{array}{rrrrrrrrrrrrrrrrrr} #1 \end{array}\right)}}

\newcommand{\Matrixc}[1]{\ensuremath{\left(\begin{array}{cccccccccccc} #1 \end{array}\right)}}



\renewcommand{\labelenumi}{\theenumi}
\newenvironment{enumeratea}%
{\begingroup
 \renewcommand{\theenumi}{\alph{enumi}}
 \renewcommand{\labelenumi}{(\theenumi)}
 \begin{enumerate}}
 {\end{enumerate}
 \endgroup}

\newcounter{help}
\renewcommand{\thehelp}{\thesection.\arabic{equation}}

%\newenvironment{equation*}%
%{\renewcommand\endequation{\eqno (\theequation)* $$}%
%   \begin{equation}}%
%   {\end{equation}\renewcommand\endequation{\eqno \@eqnnum
%$$\global\@ignoretrue}}

%\input{psfig.tex}

\author{Martin Golubitsky and Michael Dellnitz}

%\newenvironment{matlabEquation}%
%{\renewcommand\endequation{\eqno (\theequation*) $$}%
%   \begin{equation}}%
%   {\end{equation}\renewcommand\endequation{\eqno \@eqnnum
% $$\global\@ignoretrue}}

\newcommand{\soln}{\textbf{Solution:} }
\newcommand{\exercap}[1]{\centerline{Figure~\ref{#1}}}
\newcommand{\exercaptwo}[1]{\centerline{Figure~\ref{#1}a\hspace{2.1in}
Figure~\ref{#1}b}}
\newcommand{\exercapthree}[1]{\centerline{Figure~\ref{#1}a\hspace{1.2in}
Figure~\ref{#1}b\hspace{1.2in}Figure~\ref{#1}c}}
\newcommand{\para}{\hspace{0.4in}}

\usepackage{ifluatex}
\ifluatex
\ifcsname displaysolutions\endcsname%
\else
\renewenvironment{solution}{\suppress}{\endsuppress}
\fi
\else
\renewenvironment{solution}{}{}
\fi

\ifcsname answer\endcsname
\renewcommand{\answer}{}
\fi

%\ifxake
%\newenvironment{matlabEquation}{\begin{equation}}{\end{equation}}
%\else
\newenvironment{matlabEquation}%
{\let\oldtheequation\theequation\renewcommand{\theequation}{\oldtheequation*}\begin{equation}}%
  {\end{equation}\let\theequation\oldtheequation}
%\fi

\makeatother

\newcommand{\RED}[1]{{\color{red}{#1}}} 


\begin{document}

For each subset of a vector space given in Exercises~\eqref{C5.2.3A}-\eqref{C5.2.3D} determine whether the subset is a vector subspace and if it is a vector subspace, find the smallest number of vectors that spans the space.

\begin{exercise}\label{C5.2.3A}

$S = \{p(t)\in\mathcal{P}_5 : p(2) = 0 \quad\mbox{and}\quad p'(1) = 0 \}$

\begin{solution}

\ans $S$ is a subspace that is spanned by four vectors.

\soln
Verify that $S$ is a subspace by showing that it is closed under addition and scalar multiplication.  Let $p,q$ be in the subset.  Then
\[
\begin{array}{rclclcl}
(p+q)(2) & = & p(2)+q(2) & = & 0 + 0 & = & 0 \\
(p+q)'(1) & = & p'(1)+q'(1) & = & 0 + 0 & = & 0
\end{array}
\]
So the subset is closed under addition. Let $c\in\R$ and calculate
\[
(cp)(2) = cp(2) = c0 = 0  \quad\mbox{and} \quad (cp)'(1) = cp'(1) = c0 = 0
\]
Hence the subset is also closed under scalar multiplication, and the subset is a subspace.

Next we compute a spanning set for $S$.  A polynomial $p\in \mathcal{P}_5 $ has the form 
\[
p(t) = a_0 + a_1t + a_2t^2 + a_3 t^3 + a_4t^4 + a_5t^5
\]
where $a_0,a_1,a_2,a_3,a_4,a_5\in\R$.  It follows that $\dim\mathcal{P}_5 = 6$. 
We calculate the conditions on the coefficients that are needed for $p(t)$ to be in the subspace.  First note that 
\[
\begin{array}{rcl}
p'(t) & = & a_1 + 2a_2t + 3a_3 t^2 + 4a_4t^3 + 5a_5t^4\\
p(2) & = & a_0 + 2a_1 + 4a_2 + 8a_3  + 16a_4 + 32a_5 = 0\\
p'(1) & = & a_1 + 2a_2 + 3a_3 + 4a_4 + 5a_5 = 0
\end{array}
\]
Thus $p(t)$ is in the subspace if and only if
\begin{equation} \label{Cp5}
\begin{array}{rcl}
p(2) = a_0 + 2a_1 + 4a_2 + 8a_3  + 16a_4 + 32a_5 & = & 0\\
p'(1)= a_1 + 2a_2 + 3a_3 + 4a_4 + 5a_5 & = & 0
\end{array}
\end{equation}
It follows from \eqref{Cp5} that $p(t)$ is in the subspace if and only if $a_0$ and $a_1$ are determined by $a_2,a_3,a_4,a_5$.  Hence the subspace is spanned by four vectors.
\end{solution}
\end{exercise}

\begin{exercise}\label{C5.2.3B}

$T = $ symmetric $2\times 2$ matrices.  That is, $T$ is the set of $2\times 2$ matrices $A$ so that $A=A^T$.

\begin{solution}

\ans $T$ is a subspace that is spanned by $3$ vectors.

\soln
Symmetric $n\times n$ matrices form a subspace of matrices.  Suppose $A= A^T$ and $B = B^T$ are $n\times n$ matrices.  Then
\[
(A+B)^T = A^T + B^T = A + B  \quad\mbox{and} \quad (cA)^T = cA^T = cA
\]
So the set of symmetric matrices is closed under addition and scalar multiplication, and is a subspace. Specifically $2\times 2$ symmetric matrices have the form  
\[
S = \begin{bmatrix}
 a & b \\  b &c  
\end{bmatrix}
\]
and this set is spanned by $3$ vectors:
\[
\begin{bmatrix}
 1 & 0 \\  0 &0  
\end{bmatrix} 
\qquad
\begin{bmatrix}
 0 & 1 \\  1 &0  
\end{bmatrix}
\qquad
\begin{bmatrix}
 0 & 0 \\  0 & 1  
\end{bmatrix}
\]
\end{solution}
\end{exercise}

\begin{exercise}\label{C5.2.3C}
$U = 2\times 3$ matrices in reduced row-echelon form

\begin{solution}

\ans $U$ is not a subspace.

\soln
The reduced echelon form matrices $E$ do not form a subspace.  Suppose $E$ has a pivot is the $i,j$ position.  Then the $(i,j)^{th}$ entry of $E$ is 1.  But $2E$ does not have a 1 in the $(i,j)^{th}$ entry and the set is not closed under scalar multiplication.
\end{solution}
\end{exercise}

\begin{exercise}\label{C5.2.3D}
Let $A$ be the $3\times 4$ matrix
\[
A = \Matrix{ 1 & 2 & 1 & 2\\ 1 & 1 & 0 & 1 \\ 0 & 1 & 1 & 1}
\] 
and let 
\[
V = \{y\in\R^3: \mbox{there exists }  x\in\R^4 \mbox{ such that } Ax = y\}
\]


\begin{solution}
\ans $V$ is a subspace that is spanned by $2$ vectors.

\soln
To show that $V$ is a subspace of $\R^3$, let $y_1, y_2$ be vectors in $V$; that is, $y_1,y_2\in\R^3$ such that there exists vectors $x_1,x_2\in\R^4$ satisfying $Ax_1 = y_1$ and $Ax_2 = y_2$.  Then 
\[
A(x_1+x_2) = Ax_1 + Ax_2 = y_1 + y_2
\] 
and $y_1+y_2 \in V$.  Similarly, $cy_1\in V$.  To verify calculate
\[
A(cx_1) = cAx_1 = c y_1.
\]
So $V$ is closed under addition and scalar multiplication. 

The vector subspace $V$ is spanned by the four columns of $A$
\[
w_1 = \Matrix{1 \\ 1 \\ 0} \quad  w_2 = \Matrix{2 \\ 1 \\ 1} \quad  w_3 = \Matrix{1 \\ 0 \\ 1} \quad  w_4 = \Matrix{2 \\ 1 \\ 1} 
\]
Since $w_4 = w_2$ and $w_3 = w_2 - w_1$, it follows that $V$ is spanned by two vectors $w_1$ and $w_2$.  Since $V$ cannot be spanned by one vector, it follows that $V$ is spanned by exactly $2$ vectors.
\end{solution}
\end{exercise}

  

\end{document}
