\documentclass{ximera}

\usepackage{epsfig}

\graphicspath{
  {./}
  {figures/}
}

\usepackage{epstopdf}
%\usepackage{ulem}
\usepackage[normalem]{ulem}

\epstopdfsetup{outdir=./}

\usepackage{morewrites}
\makeatletter
\newcommand\subfile[1]{%
\renewcommand{\input}[1]{}%
\begingroup\skip@preamble\otherinput{#1}\endgroup\par\vspace{\topsep}
\let\input\otherinput}
\makeatother

\newcommand{\EXER}{}
\newcommand{\includeexercises}{\EXER\directlua{dofile(kpse.find_file("exercises","lua"))}}

\newenvironment{computerExercise}{\begin{exercise}}{\end{exercise}}

%\newcounter{ccounter}
%\setcounter{ccounter}{1}
%\newcommand{\Chapter}[1]{\setcounter{chapter}{\arabic{ccounter}}\chapter{#1}\addtocounter{ccounter}{1}}

%\newcommand{\section}[1]{\section{#1}\setcounter{thm}{0}\setcounter{equation}{0}}

%\renewcommand{\theequation}{\arabic{chapter}.\arabic{section}.\arabic{equation}}
%\renewcommand{\thefigure}{\arabic{chapter}.\arabic{figure}}
%\renewcommand{\thetable}{\arabic{chapter}.\arabic{table}}

%\newcommand{\Sec}[2]{\section{#1}\markright{\arabic{ccounter}.\arabic{section}.#2}\setcounter{equation}{0}\setcounter{thm}{0}\setcounter{figure}{0}}
  
\newcommand{\Sec}[2]{\section{#1}}

\setcounter{secnumdepth}{2}
%\setcounter{secnumdepth}{1} 

%\newcounter{THM}
%\renewcommand{\theTHM}{\arabic{chapter}.\arabic{section}}

\newcommand{\trademark}{{R\!\!\!\!\!\bigcirc}}
%\newtheorem{exercise}{}

\newcommand{\dfield}{{\sf dfield9}}
\newcommand{\pplane}{{\sf pplane10}}
\newcommand{\PPLANE}{{\sf PPLANE10}}

% BADBAD: \newcommand{\Bbb}{\bf}. % Package amsfonts Warning: Obsolete command \Bbb; \mathbb should be used instead.

\newcommand{\R}{\mbox{$\mathbb{R}$}}
\let\C\relax
\newcommand{\C}{\mbox{$\mathbb{C}$}}
\newcommand{\Z}{\mbox{$\mathbb{Z}$}}
\newcommand{\N}{\mbox{$\mathbb{N}$}}
\newcommand{\D}{\mbox{{\bf D}}}
\usepackage{amssymb}
%\newcommand{\qed}{\hfill\mbox{\raggedright$\square$} \vspace{1ex}}
%\newcommand{\proof}{\noindent {\bf Proof:} \hspace{0.1in}}

\newcommand{\setmin}{\;\mbox{--}\;}
\newcommand{\Matlab}{{M\small{AT\-LAB}} }
\newcommand{\Matlabp}{{M\small{AT\-LAB}}}
\newcommand{\computer}{\Matlab Instructions}
\renewcommand{\computer}{M\small{ATLAB} Instructions}
\newcommand{\half}{\mbox{$\frac{1}{2}$}}
\newcommand{\compose}{\raisebox{.15ex}{\mbox{{\scriptsize$\circ$}}}}
\newcommand{\AND}{\quad\mbox{and}\quad}
\newcommand{\vect}[2]{\left(\begin{array}{c} #1_1 \\ \vdots \\
 #1_{#2}\end{array}\right)}
\newcommand{\mattwo}[4]{\left(\begin{array}{rr} #1 & #2\\ #3
&#4\end{array}\right)}
\newcommand{\mattwoc}[4]{\left(\begin{array}{cc} #1 & #2\\ #3
&#4\end{array}\right)}
\newcommand{\vectwo}[2]{\left(\begin{array}{r} #1 \\ #2\end{array}\right)}
\newcommand{\vectwoc}[2]{\left(\begin{array}{c} #1 \\ #2\end{array}\right)}

\newcommand{\ignore}[1]{}


\newcommand{\inv}{^{-1}}
\newcommand{\CC}{{\cal C}}
\newcommand{\CCone}{\CC^1}
\newcommand{\Span}{{\rm span}}
\newcommand{\rank}{{\rm rank}}
\newcommand{\trace}{{\rm tr}}
\newcommand{\RE}{{\rm Re}}
\newcommand{\IM}{{\rm Im}}
\newcommand{\nulls}{{\rm null\;space}}

\newcommand{\dps}{\displaystyle}
\newcommand{\arraystart}{\renewcommand{\arraystretch}{1.8}}
\newcommand{\arrayfinish}{\renewcommand{\arraystretch}{1.2}}
\newcommand{\Start}[1]{\vspace{0.08in}\noindent {\bf Section~\ref{#1}}}
\newcommand{\exer}[1]{\noindent {\bf \ref{#1}}}
\newcommand{\ans}{\textbf{Answer:} }
\newcommand{\matthree}[9]{\left(\begin{array}{rrr} #1 & #2 & #3 \\ #4 & #5 & #6
\\ #7 & #8 & #9\end{array}\right)}
\newcommand{\cvectwo}[2]{\left(\begin{array}{c} #1 \\ #2\end{array}\right)}
\newcommand{\cmatthree}[9]{\left(\begin{array}{ccc} #1 & #2 & #3 \\ #4 & #5 &
#6 \\ #7 & #8 & #9\end{array}\right)}
\newcommand{\vecthree}[3]{\left(\begin{array}{r} #1 \\ #2 \\
#3\end{array}\right)}
\newcommand{\cvecthree}[3]{\left(\begin{array}{c} #1 \\ #2 \\
#3\end{array}\right)}
\newcommand{\cmattwo}[4]{\left(\begin{array}{cc} #1 & #2\\ #3
&#4\end{array}\right)}

\newcommand{\Matrix}[1]{\ensuremath{\left(\begin{array}{rrrrrrrrrrrrrrrrrr} #1 \end{array}\right)}}

\newcommand{\Matrixc}[1]{\ensuremath{\left(\begin{array}{cccccccccccc} #1 \end{array}\right)}}



\renewcommand{\labelenumi}{\theenumi}
\newenvironment{enumeratea}%
{\begingroup
 \renewcommand{\theenumi}{\alph{enumi}}
 \renewcommand{\labelenumi}{(\theenumi)}
 \begin{enumerate}}
 {\end{enumerate}
 \endgroup}

\newcounter{help}
\renewcommand{\thehelp}{\thesection.\arabic{equation}}

%\newenvironment{equation*}%
%{\renewcommand\endequation{\eqno (\theequation)* $$}%
%   \begin{equation}}%
%   {\end{equation}\renewcommand\endequation{\eqno \@eqnnum
%$$\global\@ignoretrue}}

%\input{psfig.tex}

\author{Martin Golubitsky and Michael Dellnitz}

%\newenvironment{matlabEquation}%
%{\renewcommand\endequation{\eqno (\theequation*) $$}%
%   \begin{equation}}%
%   {\end{equation}\renewcommand\endequation{\eqno \@eqnnum
% $$\global\@ignoretrue}}

\newcommand{\soln}{\textbf{Solution:} }
\newcommand{\exercap}[1]{\centerline{Figure~\ref{#1}}}
\newcommand{\exercaptwo}[1]{\centerline{Figure~\ref{#1}a\hspace{2.1in}
Figure~\ref{#1}b}}
\newcommand{\exercapthree}[1]{\centerline{Figure~\ref{#1}a\hspace{1.2in}
Figure~\ref{#1}b\hspace{1.2in}Figure~\ref{#1}c}}
\newcommand{\para}{\hspace{0.4in}}

\usepackage{ifluatex}
\ifluatex
\ifcsname displaysolutions\endcsname%
\else
\renewenvironment{solution}{\suppress}{\endsuppress}
\fi
\else
\renewenvironment{solution}{}{}
\fi

\ifcsname answer\endcsname
\renewcommand{\answer}{}
\fi

%\ifxake
%\newenvironment{matlabEquation}{\begin{equation}}{\end{equation}}
%\else
\newenvironment{matlabEquation}%
{\let\oldtheequation\theequation\renewcommand{\theequation}{\oldtheequation*}\begin{equation}}%
  {\end{equation}\let\theequation\oldtheequation}
%\fi

\makeatother

\newcommand{\RED}[1]{{\color{red}{#1}}} 


\title{The Proof of the Main Theorem}

\begin{document}
\begin{abstract}
\end{abstract}
\maketitle

 \label{S:5.6}

We begin the proof of Theorem~\ref{basis=span+indep} with two
lemmas on linearly independent and spanning sets.

\begin{lemma}  \label{reducetoindep}
Let $\{w_1,\ldots,w_k\}$ be a set of vectors in a vector space
$V$ and let $W$ be the subspace spanned by these vectors.  Then
there is a linearly independent subset of $\{w_1,\ldots,w_k\}$
that also spans $W$.
\end{lemma}\index{linearly!independent}

\begin{proof} If $\{w_1,\ldots,w_k\}$ is linearly independent, then the
lemma is proved.  If not, then the set $\{w_1,\ldots,w_k\}$ is
linearly dependent.  If this set is linearly dependent, then at
least one of the vectors is a linear combination of the others.
By renumbering if necessary, we can assume that $w_k$ is a
linear combination of $w_1,\ldots,w_{k-1}$; that is,
\[
w_k = a_1w_1 + \cdots + a_{k-1}w_{k-1}.
\]
Now suppose that $w\in W$.  Then
\[
w = b_1w_1 + \cdots + b_kw_k.
\]
It follows that
\[
w = (b_1+b_ka_1)w_1 + \cdots + (b_{k-1}+b_ka_{k-1})w_{k-1},
\]
and that $W=\Span\{w_1,\ldots,w_{k-1}\}$.  If the vectors
$w_1,\ldots,w_{k-1}$ are linearly independent, then the proof of
the lemma is complete.  If not, continue inductively until a
linearly independent subset of the $w_j$ that also spans $W$ is
found.  \end{proof}

The important point in proving that linear independence together
with spanning imply that we have a basis is discussed in the next
lemma.

\begin{lemma}  \label{lem:lindep}
Let $W$ be an $m$-dimensional vector space and let $k>m$ be an integer.
Then any set of $k$ vectors in $W$ is linearly dependent.
\end{lemma} \index{linearly!dependent}

\begin{proof} Since the dimension of $W$ is $m$ we know that
this vector space can be written as $W=\Span\{v_1,\ldots,v_m\}$.
Moreover, Lemma~\ref{reducetoindep} implies that the vectors
$v_1,\ldots,v_m$ are linearly independent.  Suppose that
$\{w_1,\ldots,w_k\}$ is another set of vectors where $k>m$.
We have to show that the vectors $w_1,\ldots,w_k$ are linearly
dependent; that is, we must show that there exist scalars
$r_1,\ldots,r_k$ not all of which are zero that satisfy
\begin{equation} \label{independence1}
r_1w_1 + \cdots + r_kw_k = 0.
\end{equation}
We find these scalars by solving a system of linear equations, as
we now show.

The fact that $W$ is spanned by the vectors $v_j$ implies that
\begin{eqnarray*}
w_1 & = & a_{11}v_1 + \cdots + a_{m1}v_m\ \\
w_2 & = & a_{12}v_1 + \cdots + a_{m2}v_m\ \\
 & \vdots & \\\
w_k & = & a_{1k}v_1 + \cdots + a_{mk}v_m.
\end{eqnarray*}
It follows that $r_1w_1 + \cdots + r_kw_k$ equals
\[
\begin{array}{ll}
r_1(a_{11}v_1 + \cdots + a_{m1}v_m) & +  \\
r_2(a_{12}v_1 + \cdots + a_{m2}v_m) & + \cdots + \\
r_k(a_{1k}v_1 + \cdots + a_{mk}v_m) &
\end{array}
\]
Rearranging terms leads to the expression:
\begin{equation}   \label{e:r1v1etc}
\begin{array}{ll}
(a_{11}r_1 + \cdots + a_{1k}r_k)v_1\ & + \\
(a_{21}r_1 + \cdots + a_{2k}r_k)v_2\ & + \cdots + \\
(a_{m1}r_1 + \cdots + a_{mk}r_k)v_m. &
\end{array}
\end{equation}
Thus, \eqref{independence1} is valid if and only if \eqref{e:r1v1etc}
sums to zero.  Since the set $\{v_1,\ldots,v_m\}$ is linearly
independent, \eqref{e:r1v1etc} can equal zero if and only if
\begin{eqnarray*}
a_{11}r_1 + \cdots + a_{1k}r_k & = & 0\ \\
a_{21}r_1 + \cdots + a_{2k}r_k & = & 0\ \\
          & \vdots &   \\
a_{m1}r_1 + \cdots + a_{mk}r_k & = & 0.
\end{eqnarray*}
Since $m<k$, Chapter~\ref{lineq}, Theorem~\ref{number} implies that
this system of homogeneous linear equations always has a nonzero
solution $r=(r_1,\ldots,r_k)$ --- from which it follows that the
$w_i$ are linearly dependent.  \end{proof}

\begin{corollary}  \label{basis<n}
Let $V$ be a vector space of dimension $n$ and let $\{u_1,\ldots,u_k\}$ be a 
linearly independent set of vectors in $V$.  Then $k\le n$.
\end{corollary} \index{linearly!independent}
\index{dimension}\index{vector!space}

\begin{proof} If $k>n$ then Lemma~\ref{lem:lindep} implies that
$\{u_1,\ldots,u_k\}$ is linearly dependent.  Since we have
assumed that this set is linearly independent, it follows that
$k\leq n$. \end{proof}

\vspace{0.1in}

\begin{proof}[Proof of Theorem~\ref{basis=span+indep}]
Suppose that ${\cal B}=\{w_1,\ldots,w_k\}$ is a basis for $W$.
By definition, ${\cal B}$ spans $W$ and $k=\dim W$.  We must
show that ${\cal B}$ is linearly independent.  Suppose ${\cal B}$
is linearly dependent, then Lemma~\ref{reducetoindep} implies
that there is a proper subset of ${\cal B}$ that spans $W$ (and
is linearly independent). This contradicts the fact that as a
basis ${\cal B}$ has the smallest number of elements of any
spanning set for $W$.

Suppose that ${\cal B}=\{w_1,\ldots,w_k\}$ both spans $W$ and is
linearly independent.  Linear independence and Corollary~\ref{basis<n}
imply that $k\leq\dim W$.  Since, by definition, any spanning set
of $W$ has at least $\dim W$ vectors, it follows that $k\geq\dim W$.
Thus, $k = \dim W$ and  ${\cal B}$ is a basis.   \end{proof}

\subsection*{Extending Linearly Independent Sets to Bases} \index{basis}

Lemma~\ref{reducetoindep} leads to one approach to finding
bases.  Suppose that the subspace $W$ is spanned by a finite
set of vectors $\{w_1,\ldots,w_k\}$.  Then, we can throw out
vectors one by one until we arrive at a linearly independent
subset of the $w_j$.  This subset is a basis for $W$.

We now discuss a second approach to finding a basis for a
nonzero subspace $W$ of a finite dimensional vector space $V$.

\begin{lemma}  \label{extendindep}
Let $\{u_1,\ldots,u_k\}$ be a linearly independent set of
vectors in a vector space $V$ and assume that
\[
u_{k+1}\not\in\Span\{u_1,\ldots,u_k\}.
\]
Then $\{u_1,\ldots,u_{k+1}\}$ is also a linearly independent
set.
\end{lemma} \index{linearly!independent}

\begin{proof}  Let $r_1,\ldots,r_{k+1}$ be scalars such that
\begin{equation}  \label{rk+1}
r_1u_1 + \cdots + r_{k+1}u_{k+1} = 0.
\end{equation}
To prove independence, we need to show that all $r_j=0$.
Suppose $r_{k+1}\neq 0$.  Then we can solve \eqref{rk+1} for
\[
u_{k+1} = -\frac{1}{r_{k+1}}(r_1u_1+\cdots +r_ku_k),
\]
which implies that $u_{k+1}\in \Span\{u_1,\ldots,u_k\}$.  This
contradicts the choice of $u_{k+1}$.  So $r_{k+1}=0$ and
\[
r_1u_1 + \cdots + r_ku_k = 0.
\]
Since $\{u_1,\ldots,u_k\}$ is linearly independent, it follows
that $r_1=\cdots =r_k=0$.  \end{proof}

The second method for constructing a basis is:
\index{basis!construction}
\begin{itemize}
\item        Choose a nonzero vector $w_1$ in $W$.
\item If $W$ is not spanned by $w_1$, then choose a vector $w_2$
that is not on the line spanned by $w_1$.
\item        If $W\neq\Span\{w_1,w_2\}$, then choose a vector
$w_3\not\in
\Span\{w_1,w_2\}$.
\item        If $W\neq\Span\{w_1,w_2,w_3\}$, then choose a vector
$w_4\not\in
\Span\{w_1,w_2,w_3\}$.
\item Continue until a spanning set\index{spanning set} for $W$ is
found.  This set is a basis for $W$.
\end{itemize}

We now justify this approach to finding bases for subspaces.
Suppose that $W$ is a subspace of a finite dimensional vector
space $V$.  For example, suppose that $W\subset\R^n$. Then
our approach to finding a basis of $W$ is as follows.  Choose a
nonzero vector $w_1\in W$.  If $W=\Span\{w_1\}$, then we are
done.  If not, choose a vector $w_2\in W\setmin\Span\{w_1\}$.
It follows from Lemma~\ref{extendindep} that $\{w_1,w_2\}$ is
linearly independent.  If $W=\Span\{w_1,w_2\}$, then
Theorem~\ref{basis=span+indep} implies that $\{w_1,w_2\}$ is
a basis for $W$, $\dim W=2$, and we are done.  If not, choose
$w_3\in W\setmin\Span\{w_1,w_2\}$ and $\{w_1,w_2,w_3\}$ is
linearly independent.  The finite dimension of $V$ implies that
continuing inductively must lead to a spanning set of linear
independent vectors for $W$ --- which by
Theorem~\ref{basis=span+indep} is a basis. This discussion proves:
\begin{corollary}  \label{c:extendindependent}
Every linearly independent subset of a finite dimensional vector
space $V$ can be extended to a basis of $V$.
\end{corollary}


\subsection*{Further consequences of Theorem~\ref{basis=span+indep}}

We summarize here several important facts about dimensions.

\begin{corollary}  \label{dimensiondecreases}
Let $W$ be a subspace of a finite dimensional vector space $V$.
\begin{itemize}
\item[(a)]   Suppose that $W$ is a proper subspace\index{subspace!proper}.
Then $\dim W < \dim V$\index{dimension}.
\item[(b)]   Suppose that $\dim W = \dim V$.  Then $W=V$.
\end{itemize}
\end{corollary}

\begin{proof}
(a) Let $\dim W = k$ and let $\{w_1,\ldots,w_k\}$ be a basis for
$W$.  Since $W$ is a proper subspace of $V$, there is a vector
$w\in V\setmin W$.  It follows from Lemma~\ref{extendindep} that
$\{w_1,\ldots,w_k,w\}$ is a linearly independent set.  Therefore,
Corollary~\ref{basis<n} implies that $k+1\le n$.

(b) Let $\{w_1,\ldots,w_k\}$ be a basis for $W$.
Theorem~\ref{basis=span+indep} implies that this set is linearly
independent.  If $\{w_1,\ldots,w_k\}$ does not span $V$, then it
can be extended to a basis as above.  But then $\dim V > \dim W$,
which is a contradiction.  \end{proof}

\begin{corollary} \label{C:dim=n}
Let ${\cal B}=\{w_1,\ldots,w_n\}$ be a set of $n$ vectors in an
$n$-dimensional vector space $V$.  Then the following are equivalent:
\begin{itemize}
\item[(a)]  ${\cal B}$ is a spanning set of $V$, \index{span}
\item[(b)]  ${\cal B}$ is a basis for $V$, and \index{basis}
\item[(c)] ${\cal B}$ is a linearly independent set.
\index{linearly!independent}
\end{itemize}
\end{corollary}

\begin{proof}    By definition, (a) implies (b) since a basis is a spanning
set with the number of vectors equal to the dimension of the space.
Theorem~\ref{basis=span+indep} states that a basis is a linearly
independent set; so (b) implies (c). If ${\cal B}$ is a linearly
independent set of $n$ vectors, then it spans a subspace $W$ of
dimension $n$.  It follows from Corollary~\ref{dimensiondecreases}(b)
that $W=V$ and that (c) implies (a).  \end{proof}


\subsubsection*{Subspaces of $\R^3$}
\index{$\R^3$!subspaces}

We can now classify all subspaces of $\R^3$.  They are:  the origin, lines
through the origin, planes through the origin, and $\R^3$.  All of these
sets were shown to be subspaces in Example~\ref{EX:subspaces}(a--c).

To verify that these sets are the only subspaces of $\R^3$, note that
Theorem~\ref{basis=span+indep} implies that proper subspaces of $\R^3$ have
dimension equal either to one or two. (The zero dimensional subspace is the
origin and the only three dimensional subspace is $\R^3$ itself.)  One
dimensional subspaces of $\R^3$ are spanned by one nonzero vector and are just
lines through the origin.  See Example~\ref{EX:subspaces}(b).  We claim that
all two dimensional subspaces are planes through the origin.

Suppose that $W\subset\R^3$ is a subspace spanned by two non-collinear vectors
$w_1$ and $w_2$.\index{collinear}  We show that $W$ is a plane \index{plane}
through the origin using results in Chapter~\ref{lineq}.  Observe that there
is a vector $N=(N_1,N_2,N_3)$ perpendicular to $w_1=(a_{11},a_{12},a_{13})$
and $w_2=(a_{21},a_{22},a_{23})$.  Such a vector $N$ satisfies the two linear
equations:
\begin{eqnarray*}
w_1\cdot N & = & a_{11}N_1 + a_{12}N_2 + a_{13}N_3 = 0 \\
w_2\cdot N & = & a_{21}N_1 + a_{22}N_2 + a_{23}N_3 = 0.
\end{eqnarray*}
Chapter~\ref{lineq}, Theorem~\ref{number} implies that a system of two linear
equations in three unknowns has a nonzero solution.  Let $P$ be the plane
perpendicular \index{perpendicular} to $N$ that contains the origin.  We show
that $W=P$ and hence that the claim is valid.

The choice of $N$ shows that the vectors $w_1$ and $w_2$ are both in $P$. In
fact, since $P$ is a subspace it contains every vector in $\Span\{w_1,w_2\}$.
Thus $W\subset P$.  If $P$ contains just one additional vector $w_3\in\R^3$
that is not in $W$, then the span of $w_1,w_2,w_3$ is three dimensional and
$P=W=\R^3$.





\includeexercises













\end{document}
