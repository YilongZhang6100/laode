\documentclass{ximera}

\author{Marty Golubitsky}

\usepackage{epsfig}

\graphicspath{
  {./}
  {figures/}
}

\usepackage{epstopdf}
%\usepackage{ulem}
\usepackage[normalem]{ulem}

\epstopdfsetup{outdir=./}

\usepackage{morewrites}
\makeatletter
\newcommand\subfile[1]{%
\renewcommand{\input}[1]{}%
\begingroup\skip@preamble\otherinput{#1}\endgroup\par\vspace{\topsep}
\let\input\otherinput}
\makeatother

\newcommand{\EXER}{}
\newcommand{\includeexercises}{\EXER\directlua{dofile(kpse.find_file("exercises","lua"))}}

\newenvironment{computerExercise}{\begin{exercise}}{\end{exercise}}

%\newcounter{ccounter}
%\setcounter{ccounter}{1}
%\newcommand{\Chapter}[1]{\setcounter{chapter}{\arabic{ccounter}}\chapter{#1}\addtocounter{ccounter}{1}}

%\newcommand{\section}[1]{\section{#1}\setcounter{thm}{0}\setcounter{equation}{0}}

%\renewcommand{\theequation}{\arabic{chapter}.\arabic{section}.\arabic{equation}}
%\renewcommand{\thefigure}{\arabic{chapter}.\arabic{figure}}
%\renewcommand{\thetable}{\arabic{chapter}.\arabic{table}}

%\newcommand{\Sec}[2]{\section{#1}\markright{\arabic{ccounter}.\arabic{section}.#2}\setcounter{equation}{0}\setcounter{thm}{0}\setcounter{figure}{0}}
  
\newcommand{\Sec}[2]{\section{#1}}

\setcounter{secnumdepth}{2}
%\setcounter{secnumdepth}{1} 

%\newcounter{THM}
%\renewcommand{\theTHM}{\arabic{chapter}.\arabic{section}}

\newcommand{\trademark}{{R\!\!\!\!\!\bigcirc}}
%\newtheorem{exercise}{}

\newcommand{\dfield}{{\sf dfield9}}
\newcommand{\pplane}{{\sf pplane10}}
\newcommand{\PPLANE}{{\sf PPLANE10}}

% BADBAD: \newcommand{\Bbb}{\bf}. % Package amsfonts Warning: Obsolete command \Bbb; \mathbb should be used instead.

\newcommand{\R}{\mbox{$\mathbb{R}$}}
\let\C\relax
\newcommand{\C}{\mbox{$\mathbb{C}$}}
\newcommand{\Z}{\mbox{$\mathbb{Z}$}}
\newcommand{\N}{\mbox{$\mathbb{N}$}}
\newcommand{\D}{\mbox{{\bf D}}}
\usepackage{amssymb}
%\newcommand{\qed}{\hfill\mbox{\raggedright$\square$} \vspace{1ex}}
%\newcommand{\proof}{\noindent {\bf Proof:} \hspace{0.1in}}

\newcommand{\setmin}{\;\mbox{--}\;}
\newcommand{\Matlab}{{M\small{AT\-LAB}} }
\newcommand{\Matlabp}{{M\small{AT\-LAB}}}
\newcommand{\computer}{\Matlab Instructions}
\renewcommand{\computer}{M\small{ATLAB} Instructions}
\newcommand{\half}{\mbox{$\frac{1}{2}$}}
\newcommand{\compose}{\raisebox{.15ex}{\mbox{{\scriptsize$\circ$}}}}
\newcommand{\AND}{\quad\mbox{and}\quad}
\newcommand{\vect}[2]{\left(\begin{array}{c} #1_1 \\ \vdots \\
 #1_{#2}\end{array}\right)}
\newcommand{\mattwo}[4]{\left(\begin{array}{rr} #1 & #2\\ #3
&#4\end{array}\right)}
\newcommand{\mattwoc}[4]{\left(\begin{array}{cc} #1 & #2\\ #3
&#4\end{array}\right)}
\newcommand{\vectwo}[2]{\left(\begin{array}{r} #1 \\ #2\end{array}\right)}
\newcommand{\vectwoc}[2]{\left(\begin{array}{c} #1 \\ #2\end{array}\right)}

\newcommand{\ignore}[1]{}


\newcommand{\inv}{^{-1}}
\newcommand{\CC}{{\cal C}}
\newcommand{\CCone}{\CC^1}
\newcommand{\Span}{{\rm span}}
\newcommand{\rank}{{\rm rank}}
\newcommand{\trace}{{\rm tr}}
\newcommand{\RE}{{\rm Re}}
\newcommand{\IM}{{\rm Im}}
\newcommand{\nulls}{{\rm null\;space}}

\newcommand{\dps}{\displaystyle}
\newcommand{\arraystart}{\renewcommand{\arraystretch}{1.8}}
\newcommand{\arrayfinish}{\renewcommand{\arraystretch}{1.2}}
\newcommand{\Start}[1]{\vspace{0.08in}\noindent {\bf Section~\ref{#1}}}
\newcommand{\exer}[1]{\noindent {\bf \ref{#1}}}
\newcommand{\ans}{\textbf{Answer:} }
\newcommand{\matthree}[9]{\left(\begin{array}{rrr} #1 & #2 & #3 \\ #4 & #5 & #6
\\ #7 & #8 & #9\end{array}\right)}
\newcommand{\cvectwo}[2]{\left(\begin{array}{c} #1 \\ #2\end{array}\right)}
\newcommand{\cmatthree}[9]{\left(\begin{array}{ccc} #1 & #2 & #3 \\ #4 & #5 &
#6 \\ #7 & #8 & #9\end{array}\right)}
\newcommand{\vecthree}[3]{\left(\begin{array}{r} #1 \\ #2 \\
#3\end{array}\right)}
\newcommand{\cvecthree}[3]{\left(\begin{array}{c} #1 \\ #2 \\
#3\end{array}\right)}
\newcommand{\cmattwo}[4]{\left(\begin{array}{cc} #1 & #2\\ #3
&#4\end{array}\right)}

\newcommand{\Matrix}[1]{\ensuremath{\left(\begin{array}{rrrrrrrrrrrrrrrrrr} #1 \end{array}\right)}}

\newcommand{\Matrixc}[1]{\ensuremath{\left(\begin{array}{cccccccccccc} #1 \end{array}\right)}}



\renewcommand{\labelenumi}{\theenumi}
\newenvironment{enumeratea}%
{\begingroup
 \renewcommand{\theenumi}{\alph{enumi}}
 \renewcommand{\labelenumi}{(\theenumi)}
 \begin{enumerate}}
 {\end{enumerate}
 \endgroup}

\newcounter{help}
\renewcommand{\thehelp}{\thesection.\arabic{equation}}

%\newenvironment{equation*}%
%{\renewcommand\endequation{\eqno (\theequation)* $$}%
%   \begin{equation}}%
%   {\end{equation}\renewcommand\endequation{\eqno \@eqnnum
%$$\global\@ignoretrue}}

%\input{psfig.tex}

\author{Martin Golubitsky and Michael Dellnitz}

%\newenvironment{matlabEquation}%
%{\renewcommand\endequation{\eqno (\theequation*) $$}%
%   \begin{equation}}%
%   {\end{equation}\renewcommand\endequation{\eqno \@eqnnum
% $$\global\@ignoretrue}}

\newcommand{\soln}{\textbf{Solution:} }
\newcommand{\exercap}[1]{\centerline{Figure~\ref{#1}}}
\newcommand{\exercaptwo}[1]{\centerline{Figure~\ref{#1}a\hspace{2.1in}
Figure~\ref{#1}b}}
\newcommand{\exercapthree}[1]{\centerline{Figure~\ref{#1}a\hspace{1.2in}
Figure~\ref{#1}b\hspace{1.2in}Figure~\ref{#1}c}}
\newcommand{\para}{\hspace{0.4in}}

\usepackage{ifluatex}
\ifluatex
\ifcsname displaysolutions\endcsname%
\else
\renewenvironment{solution}{\suppress}{\endsuppress}
\fi
\else
\renewenvironment{solution}{}{}
\fi

\ifcsname answer\endcsname
\renewcommand{\answer}{}
\fi

%\ifxake
%\newenvironment{matlabEquation}{\begin{equation}}{\end{equation}}
%\else
\newenvironment{matlabEquation}%
{\let\oldtheequation\theequation\renewcommand{\theequation}{\oldtheequation*}\begin{equation}}%
  {\end{equation}\let\theequation\oldtheequation}
%\fi

\makeatother

\newcommand{\RED}[1]{{\color{red}{#1}}} 


\begin{document}

% To which section does your exercise belong? 

\begin{exercise}\label{c10.3.10}

Let $\mathcal{S}_2$ be the set of real $2\times 2$ symmetric matrices and let $P$ be a $2\times 2$ orthogonal matrix.
\begin{enumeratea}
\item Verify that $\mathcal{S}_2$ is a 3-dimensional vector space.

\item Verify that the map $M_P: \mathcal{S}_2 \to \mathcal{S}_2$ defined by 
\[
M_P(A) = P^tAP
\]
is linear.

\item Let $P$ be the reflection matrix 
\begin{equation} \label{e:reflection_matrix}
P = \Matrix{0 & -1 \\ 1 & 0 }
\end{equation}
Verify that $P$ is an orthogonal matrix and compute the eigenvalues and eigenvectors of $M_P$.

\end{enumeratea}

\begin{solution}
\soln 
\begin{enumeratea}
\item Symmetric matrices are closed under addition and scalar multiplication.  Therefore, $\mathcal{S}_2$ is a vector space. Let  
\begin{equation} \label{e:sym_mat_base}
E_1 = \Matrixc{1 & 0 \\ 0 & 0} \quad E_2 = \Matrixc{0& 1 \\ 1 & 0} \quad E_3 = \Matrix{0 & 0 \\ 0 & 1}
\end{equation}
Observe that 
\[
\Matrixc{a & b \\ b & c} = aE_1 + bE_2 + cE_3.
\]
Hence, the vectors $\mathcal{E} = \{E_1,E_2,E_3\}$ span $\mathcal{S}_2$ and are linearly independent.  Therefore,  $\mathcal{E}$ is a basis for $\mathcal{S}_2 $ and $\dim \mathcal{S}_2 = 3$.

\item To verify linearity, compute
\[
M_P(A+B) = P^t(A+B)P = P^t(AP + BP) = P^tAP + P^tBP = M_P(A) + M_P(B)
\]
for all $A,B\in \mathcal{S}_2$ and
\[ 
M_P(cA) = P^t(cA)P = c P^tAP = cM_P(A)
\]
for all $c\in\R$.

\item Compute the matrix of $M_P$ in the basis given in \eqref{e:sym_mat_base}, where $P$ is the reflection matrix in \eqref{e:reflection_matrix}.   More precisely, compute
\begin{eqnarray*}
M_P(E_1) & = & P^t E_1P \\
 & = &    \Matrix{0 & 1 \\ -1 & 0 } \Matrixc{1 & 0 \\ 0 & 0}   \Matrix{0 & -1 \\ 1 & 0 } \\
 & = &    \Matrix{0 & 1 \\ -1 & 0 }  \Matrix{0 & -1 \\ 0 & 0 } \\
 & = & \Matrixc{0 & 0 \\ 0 & 1 } \\
 & = & E_3
\end{eqnarray*}
It follows that 
\[
[M_P(E_1)]_\mathcal{E} = \Matrixc{ 0 \\ 0 \\ 1}
\]
Similarly,
\begin{eqnarray*}
M_P(E_2) & = & -E_2\\
M_P(E_3) & = & E_1
\end{eqnarray*}
The $3\times 3$ matrix of $M_P$ in the basis $\mathcal{E}$ is 
\begin{eqnarray*}
[M_P]_\mathcal{E} & = & [ [M_P(E_1)]_\mathcal{E} | [M_P(E_2)]_\mathcal{E} |[M_P(E_3)]_\mathcal{E} ] \\
& = &  \Matrix{  0 &  0 & 1 \\  0 & -1 & 0 \\  1 & 0 & 0 }
\end{eqnarray*}
Compute the characteristic polynomial of $[M_P]_\mathcal{E}$ as
\[
p(\lambda)  = \det \Matrixc{  -\lambda & 0 &  1 \\ 0 & -1-\lambda & 0 \\ 1 &  0 & -\lambda} = (1-\lambda^2)(1+\lambda)
\]
Hence the eigenvalues of $M_P$ are $-1,-1,1$. 
The eigenvectors in $\R^3$ associated with the eigenvalue $-1$ is the null space of 
\[
\Matrix{  1 & 0 &  1 \\ 0 & 0 & 0 \\ 1 &  0 & 1} . 
\]
That space is generated by the vectors $(1,0,-1)^t$ and $(0, 1, 0)^t$.   These vectors correspond to the symmetric matrices $E_1-E_3$ and $E_2$.
The eigenvector corresponding to the eigenvalue $1$ is the null space of the matrix 
\[
\Matrix{  -1 & 0 &  1 \\ 0 & -2 & 0 \\ 1 &  0 & -1} . 
\]
That vector is $(1, 0, 1)^t$ and corresponds with the symmetric matrix $E_1-E_3$.
\end{enumeratea}

\end{solution}
\end{exercise}

\end{document}
