\documentclass{ximera}
%\usepackage{epsfig}

\graphicspath{
  {./}
  {figures/}
}

\usepackage{epstopdf}
%\usepackage{ulem}
\usepackage[normalem]{ulem}

\epstopdfsetup{outdir=./}

\usepackage{morewrites}
\makeatletter
\newcommand\subfile[1]{%
\renewcommand{\input}[1]{}%
\begingroup\skip@preamble\otherinput{#1}\endgroup\par\vspace{\topsep}
\let\input\otherinput}
\makeatother

\newcommand{\EXER}{}
\newcommand{\includeexercises}{\EXER\directlua{dofile(kpse.find_file("exercises","lua"))}}

\newenvironment{computerExercise}{\begin{exercise}}{\end{exercise}}

%\newcounter{ccounter}
%\setcounter{ccounter}{1}
%\newcommand{\Chapter}[1]{\setcounter{chapter}{\arabic{ccounter}}\chapter{#1}\addtocounter{ccounter}{1}}

%\newcommand{\section}[1]{\section{#1}\setcounter{thm}{0}\setcounter{equation}{0}}

%\renewcommand{\theequation}{\arabic{chapter}.\arabic{section}.\arabic{equation}}
%\renewcommand{\thefigure}{\arabic{chapter}.\arabic{figure}}
%\renewcommand{\thetable}{\arabic{chapter}.\arabic{table}}

%\newcommand{\Sec}[2]{\section{#1}\markright{\arabic{ccounter}.\arabic{section}.#2}\setcounter{equation}{0}\setcounter{thm}{0}\setcounter{figure}{0}}
  
\newcommand{\Sec}[2]{\section{#1}}

\setcounter{secnumdepth}{2}
%\setcounter{secnumdepth}{1} 

%\newcounter{THM}
%\renewcommand{\theTHM}{\arabic{chapter}.\arabic{section}}

\newcommand{\trademark}{{R\!\!\!\!\!\bigcirc}}
%\newtheorem{exercise}{}

\newcommand{\dfield}{{\sf dfield9}}
\newcommand{\pplane}{{\sf pplane10}}
\newcommand{\PPLANE}{{\sf PPLANE10}}

% BADBAD: \newcommand{\Bbb}{\bf}. % Package amsfonts Warning: Obsolete command \Bbb; \mathbb should be used instead.

\newcommand{\R}{\mbox{$\mathbb{R}$}}
\let\C\relax
\newcommand{\C}{\mbox{$\mathbb{C}$}}
\newcommand{\Z}{\mbox{$\mathbb{Z}$}}
\newcommand{\N}{\mbox{$\mathbb{N}$}}
\newcommand{\D}{\mbox{{\bf D}}}
\usepackage{amssymb}
%\newcommand{\qed}{\hfill\mbox{\raggedright$\square$} \vspace{1ex}}
%\newcommand{\proof}{\noindent {\bf Proof:} \hspace{0.1in}}

\newcommand{\setmin}{\;\mbox{--}\;}
\newcommand{\Matlab}{{M\small{AT\-LAB}} }
\newcommand{\Matlabp}{{M\small{AT\-LAB}}}
\newcommand{\computer}{\Matlab Instructions}
\renewcommand{\computer}{M\small{ATLAB} Instructions}
\newcommand{\half}{\mbox{$\frac{1}{2}$}}
\newcommand{\compose}{\raisebox{.15ex}{\mbox{{\scriptsize$\circ$}}}}
\newcommand{\AND}{\quad\mbox{and}\quad}
\newcommand{\vect}[2]{\left(\begin{array}{c} #1_1 \\ \vdots \\
 #1_{#2}\end{array}\right)}
\newcommand{\mattwo}[4]{\left(\begin{array}{rr} #1 & #2\\ #3
&#4\end{array}\right)}
\newcommand{\mattwoc}[4]{\left(\begin{array}{cc} #1 & #2\\ #3
&#4\end{array}\right)}
\newcommand{\vectwo}[2]{\left(\begin{array}{r} #1 \\ #2\end{array}\right)}
\newcommand{\vectwoc}[2]{\left(\begin{array}{c} #1 \\ #2\end{array}\right)}

\newcommand{\ignore}[1]{}


\newcommand{\inv}{^{-1}}
\newcommand{\CC}{{\cal C}}
\newcommand{\CCone}{\CC^1}
\newcommand{\Span}{{\rm span}}
\newcommand{\rank}{{\rm rank}}
\newcommand{\trace}{{\rm tr}}
\newcommand{\RE}{{\rm Re}}
\newcommand{\IM}{{\rm Im}}
\newcommand{\nulls}{{\rm null\;space}}

\newcommand{\dps}{\displaystyle}
\newcommand{\arraystart}{\renewcommand{\arraystretch}{1.8}}
\newcommand{\arrayfinish}{\renewcommand{\arraystretch}{1.2}}
\newcommand{\Start}[1]{\vspace{0.08in}\noindent {\bf Section~\ref{#1}}}
\newcommand{\exer}[1]{\noindent {\bf \ref{#1}}}
\newcommand{\ans}{\textbf{Answer:} }
\newcommand{\matthree}[9]{\left(\begin{array}{rrr} #1 & #2 & #3 \\ #4 & #5 & #6
\\ #7 & #8 & #9\end{array}\right)}
\newcommand{\cvectwo}[2]{\left(\begin{array}{c} #1 \\ #2\end{array}\right)}
\newcommand{\cmatthree}[9]{\left(\begin{array}{ccc} #1 & #2 & #3 \\ #4 & #5 &
#6 \\ #7 & #8 & #9\end{array}\right)}
\newcommand{\vecthree}[3]{\left(\begin{array}{r} #1 \\ #2 \\
#3\end{array}\right)}
\newcommand{\cvecthree}[3]{\left(\begin{array}{c} #1 \\ #2 \\
#3\end{array}\right)}
\newcommand{\cmattwo}[4]{\left(\begin{array}{cc} #1 & #2\\ #3
&#4\end{array}\right)}

\newcommand{\Matrix}[1]{\ensuremath{\left(\begin{array}{rrrrrrrrrrrrrrrrrr} #1 \end{array}\right)}}

\newcommand{\Matrixc}[1]{\ensuremath{\left(\begin{array}{cccccccccccc} #1 \end{array}\right)}}



\renewcommand{\labelenumi}{\theenumi}
\newenvironment{enumeratea}%
{\begingroup
 \renewcommand{\theenumi}{\alph{enumi}}
 \renewcommand{\labelenumi}{(\theenumi)}
 \begin{enumerate}}
 {\end{enumerate}
 \endgroup}

\newcounter{help}
\renewcommand{\thehelp}{\thesection.\arabic{equation}}

%\newenvironment{equation*}%
%{\renewcommand\endequation{\eqno (\theequation)* $$}%
%   \begin{equation}}%
%   {\end{equation}\renewcommand\endequation{\eqno \@eqnnum
%$$\global\@ignoretrue}}

%\input{psfig.tex}

\author{Martin Golubitsky and Michael Dellnitz}

%\newenvironment{matlabEquation}%
%{\renewcommand\endequation{\eqno (\theequation*) $$}%
%   \begin{equation}}%
%   {\end{equation}\renewcommand\endequation{\eqno \@eqnnum
% $$\global\@ignoretrue}}

\newcommand{\soln}{\textbf{Solution:} }
\newcommand{\exercap}[1]{\centerline{Figure~\ref{#1}}}
\newcommand{\exercaptwo}[1]{\centerline{Figure~\ref{#1}a\hspace{2.1in}
Figure~\ref{#1}b}}
\newcommand{\exercapthree}[1]{\centerline{Figure~\ref{#1}a\hspace{1.2in}
Figure~\ref{#1}b\hspace{1.2in}Figure~\ref{#1}c}}
\newcommand{\para}{\hspace{0.4in}}

\usepackage{ifluatex}
\ifluatex
\ifcsname displaysolutions\endcsname%
\else
\renewenvironment{solution}{\suppress}{\endsuppress}
\fi
\else
\renewenvironment{solution}{}{}
\fi

\ifcsname answer\endcsname
\renewcommand{\answer}{}
\fi

%\ifxake
%\newenvironment{matlabEquation}{\begin{equation}}{\end{equation}}
%\else
\newenvironment{matlabEquation}%
{\let\oldtheequation\theequation\renewcommand{\theequation}{\oldtheequation*}\begin{equation}}%
  {\end{equation}\let\theequation\oldtheequation}
%\fi

\makeatother

\newcommand{\RED}[1]{{\color{red}{#1}}} 

\usepackage{epsfig}

\graphicspath{
  {./}
  {figures/}
}

\usepackage{epstopdf}
%\usepackage{ulem}
\usepackage[normalem]{ulem}

\epstopdfsetup{outdir=./}

\usepackage{morewrites}
\makeatletter
\newcommand\subfile[1]{%
\renewcommand{\input}[1]{}%
\begingroup\skip@preamble\otherinput{#1}\endgroup\par\vspace{\topsep}
\let\input\otherinput}
\makeatother

\newcommand{\EXER}{}
\newcommand{\includeexercises}{\EXER\directlua{dofile(kpse.find_file("exercises","lua"))}}

\newenvironment{computerExercise}{\begin{exercise}}{\end{exercise}}

%\newcounter{ccounter}
%\setcounter{ccounter}{1}
%\newcommand{\Chapter}[1]{\setcounter{chapter}{\arabic{ccounter}}\chapter{#1}\addtocounter{ccounter}{1}}

%\newcommand{\section}[1]{\section{#1}\setcounter{thm}{0}\setcounter{equation}{0}}

%\renewcommand{\theequation}{\arabic{chapter}.\arabic{section}.\arabic{equation}}
%\renewcommand{\thefigure}{\arabic{chapter}.\arabic{figure}}
%\renewcommand{\thetable}{\arabic{chapter}.\arabic{table}}

%\newcommand{\Sec}[2]{\section{#1}\markright{\arabic{ccounter}.\arabic{section}.#2}\setcounter{equation}{0}\setcounter{thm}{0}\setcounter{figure}{0}}
  
\newcommand{\Sec}[2]{\section{#1}}

\setcounter{secnumdepth}{2}
%\setcounter{secnumdepth}{1} 

%\newcounter{THM}
%\renewcommand{\theTHM}{\arabic{chapter}.\arabic{section}}

\newcommand{\trademark}{{R\!\!\!\!\!\bigcirc}}
%\newtheorem{exercise}{}

\newcommand{\dfield}{{\sf dfield9}}
\newcommand{\pplane}{{\sf pplane10}}
\newcommand{\PPLANE}{{\sf PPLANE10}}

% BADBAD: \newcommand{\Bbb}{\bf}. % Package amsfonts Warning: Obsolete command \Bbb; \mathbb should be used instead.

\newcommand{\R}{\mbox{$\mathbb{R}$}}
\let\C\relax
\newcommand{\C}{\mbox{$\mathbb{C}$}}
\newcommand{\Z}{\mbox{$\mathbb{Z}$}}
\newcommand{\N}{\mbox{$\mathbb{N}$}}
\newcommand{\D}{\mbox{{\bf D}}}
\usepackage{amssymb}
%\newcommand{\qed}{\hfill\mbox{\raggedright$\square$} \vspace{1ex}}
%\newcommand{\proof}{\noindent {\bf Proof:} \hspace{0.1in}}

\newcommand{\setmin}{\;\mbox{--}\;}
\newcommand{\Matlab}{{M\small{AT\-LAB}} }
\newcommand{\Matlabp}{{M\small{AT\-LAB}}}
\newcommand{\computer}{\Matlab Instructions}
\renewcommand{\computer}{M\small{ATLAB} Instructions}
\newcommand{\half}{\mbox{$\frac{1}{2}$}}
\newcommand{\compose}{\raisebox{.15ex}{\mbox{{\scriptsize$\circ$}}}}
\newcommand{\AND}{\quad\mbox{and}\quad}
\newcommand{\vect}[2]{\left(\begin{array}{c} #1_1 \\ \vdots \\
 #1_{#2}\end{array}\right)}
\newcommand{\mattwo}[4]{\left(\begin{array}{rr} #1 & #2\\ #3
&#4\end{array}\right)}
\newcommand{\mattwoc}[4]{\left(\begin{array}{cc} #1 & #2\\ #3
&#4\end{array}\right)}
\newcommand{\vectwo}[2]{\left(\begin{array}{r} #1 \\ #2\end{array}\right)}
\newcommand{\vectwoc}[2]{\left(\begin{array}{c} #1 \\ #2\end{array}\right)}

\newcommand{\ignore}[1]{}


\newcommand{\inv}{^{-1}}
\newcommand{\CC}{{\cal C}}
\newcommand{\CCone}{\CC^1}
\newcommand{\Span}{{\rm span}}
\newcommand{\rank}{{\rm rank}}
\newcommand{\trace}{{\rm tr}}
\newcommand{\RE}{{\rm Re}}
\newcommand{\IM}{{\rm Im}}
\newcommand{\nulls}{{\rm null\;space}}

\newcommand{\dps}{\displaystyle}
\newcommand{\arraystart}{\renewcommand{\arraystretch}{1.8}}
\newcommand{\arrayfinish}{\renewcommand{\arraystretch}{1.2}}
\newcommand{\Start}[1]{\vspace{0.08in}\noindent {\bf Section~\ref{#1}}}
\newcommand{\exer}[1]{\noindent {\bf \ref{#1}}}
\newcommand{\ans}{\textbf{Answer:} }
\newcommand{\matthree}[9]{\left(\begin{array}{rrr} #1 & #2 & #3 \\ #4 & #5 & #6
\\ #7 & #8 & #9\end{array}\right)}
\newcommand{\cvectwo}[2]{\left(\begin{array}{c} #1 \\ #2\end{array}\right)}
\newcommand{\cmatthree}[9]{\left(\begin{array}{ccc} #1 & #2 & #3 \\ #4 & #5 &
#6 \\ #7 & #8 & #9\end{array}\right)}
\newcommand{\vecthree}[3]{\left(\begin{array}{r} #1 \\ #2 \\
#3\end{array}\right)}
\newcommand{\cvecthree}[3]{\left(\begin{array}{c} #1 \\ #2 \\
#3\end{array}\right)}
\newcommand{\cmattwo}[4]{\left(\begin{array}{cc} #1 & #2\\ #3
&#4\end{array}\right)}

\newcommand{\Matrix}[1]{\ensuremath{\left(\begin{array}{rrrrrrrrrrrrrrrrrr} #1 \end{array}\right)}}

\newcommand{\Matrixc}[1]{\ensuremath{\left(\begin{array}{cccccccccccc} #1 \end{array}\right)}}



\renewcommand{\labelenumi}{\theenumi}
\newenvironment{enumeratea}%
{\begingroup
 \renewcommand{\theenumi}{\alph{enumi}}
 \renewcommand{\labelenumi}{(\theenumi)}
 \begin{enumerate}}
 {\end{enumerate}
 \endgroup}

\newcounter{help}
\renewcommand{\thehelp}{\thesection.\arabic{equation}}

%\newenvironment{equation*}%
%{\renewcommand\endequation{\eqno (\theequation)* $$}%
%   \begin{equation}}%
%   {\end{equation}\renewcommand\endequation{\eqno \@eqnnum
%$$\global\@ignoretrue}}

%\input{psfig.tex}

\author{Martin Golubitsky and Michael Dellnitz}

%\newenvironment{matlabEquation}%
%{\renewcommand\endequation{\eqno (\theequation*) $$}%
%   \begin{equation}}%
%   {\end{equation}\renewcommand\endequation{\eqno \@eqnnum
% $$\global\@ignoretrue}}

\newcommand{\soln}{\textbf{Solution:} }
\newcommand{\exercap}[1]{\centerline{Figure~\ref{#1}}}
\newcommand{\exercaptwo}[1]{\centerline{Figure~\ref{#1}a\hspace{2.1in}
Figure~\ref{#1}b}}
\newcommand{\exercapthree}[1]{\centerline{Figure~\ref{#1}a\hspace{1.2in}
Figure~\ref{#1}b\hspace{1.2in}Figure~\ref{#1}c}}
\newcommand{\para}{\hspace{0.4in}}

\usepackage{ifluatex}
\ifluatex
\ifcsname displaysolutions\endcsname%
\else
\renewenvironment{solution}{\suppress}{\endsuppress}
\fi
\else
\renewenvironment{solution}{}{}
\fi

\ifcsname answer\endcsname
\renewcommand{\answer}{}
\fi

%\ifxake
%\newenvironment{matlabEquation}{\begin{equation}}{\end{equation}}
%\else
\newenvironment{matlabEquation}%
{\let\oldtheequation\theequation\renewcommand{\theequation}{\oldtheequation*}\begin{equation}}%
  {\end{equation}\let\theequation\oldtheequation}
%\fi

\makeatother

\newcommand{\RED}[1]{{\color{red}{#1}}} 

\begin{document}




\begin{exercise} \label{YZ_3.4.1}

Suppose $A$ is a $3\times 3$ full rank matrix. Determine how many solutions the homogeneous system $Ax=0$ has and how many solutions the inhomogenous system $Ax=b$ has.


\begin{solution}
\ans
If $A$ has full rank, the reduced echelon form of $A$ is the identity matrix $I_3$. Since both $I_3x=0$ and $I_3x=b$ have unique solutions ($x=0$ and $x=b$ respective), so \RED{do} $Ax=0$ and $Ax=b$.

\end{solution}
\end{exercise}


\begin{exercise} \label{YZ_3.4.2}
Suppose $A$ is a $3\times 3$ matrix with rank less than $3$. Determine how many solution the homogeneous system $Ax=0$ has and how many solutions the inhomogenous system $Ax=b$ has.


\begin{solution}
\ans
If $A$ has rank lower than $3$. $Ax=0$ has infinitely many solutions. $Ax=b$ can have infinitely many solutions or no solution depending on whether the system of equations is consistent or \RED{inconsistent}.

\end{solution}
\end{exercise}


\begin{exercise} \label{YZ_3.4.3}
Let $A$ be a $3\times 3$ matrix.  Suppose 
\[
A\Matrix{-1\\2\\1}=\Matrix{3\\1\\1} \textup{\quad and \quad} A\Matrix{0\\4\\0}=\Matrix{-2\\0\\1}.
\]

\begin{enumeratea}
\item Find a solution to the inhomogeneous system 
\[
Ax=\Matrix{1 \\ 1 \\2}.
\]
\item Is the solution unique?
\end{enumeratea}


\begin{solution}
\ans 

\begin{enumeratea}
\item $x=\Matrix{-1 \\ 6 \\ 1}$ is a solution. 

\item If $A$ has full rank, then the solution is unique. Otherwise, there are infinitely many solutions.
\end{enumeratea}




\soln 
\begin{enumeratea}
\item  \RED{Since} 
\[
\Matrix{3 \\ 1\\ 1}+ \Matrix{-2 \\ 0 \\ 1}=\Matrix{1 \\ 1 \\ 2}
\].

Then by linearity, 

\[
\Matrix{1 \\ 1\\ 2}=\Matrix{3 \\ 1\\ 1}+ \Matrix{-2 \\ 0 \\ 1}=A\Matrix{-1 \\ 2\\ 1}+A\Matrix{0 \\ 4 \\ 0}=A\Matrix{-1 \\ 6 \\ 1}.
\] 

So $x=\Matrix{-1 \\ 6 \\ 1}$ is a solution.
\end{enumeratea}

\end{solution}
\end{exercise}



\begin{exercise} \label{YZ_3.4.4}
Let $A$ be a $n\times m$ matrix.  Suppose there are $n$ vectors $u_1,\ldots,u_n$ such that 
\[
Au_1=\Matrixc{ 1 \\ 0 \\ \vdots \\ 0}, Au_2=\Matrixc{ 0 \\ 1 \\ \vdots \\ 0}, \ldots, Au_n=\Matrixc{0\\ 0 \\ \vdots \\ 1} .
\]
Then verify that for any $m\times 1$ vector $b$, the inhomogeneous equation $Ax=b$ always has a solution.

\begin{solution}
\soln 
Note that 
\[
b = \Matrixc{b_1 \\ b_2 \\ \vdots \\ b_n} = 
b_1\Matrixc{1 \\ 0 \\ \vdots \\ 0}+b_2\Matrixc{ 0 \\ 1 \\ \vdots \\ 0}+\RED{\cdots}+b_n\Matrixc{ 0 \\ 0 \\ \vdots \\ 1}.
\]
Therefore
\[
b = \RED{b_1}Au_1 +  \RED{b_2}Au_2 + \cdots +  \RED{b_n}Au_n=
A( \RED{b_1}u_1 +  \RED{b_2}u_2+\cdots +  \RED{b_n}u_n).
\]
\end{solution}
\end{exercise}


\begin{exercise} \label{YZ_3.4.5}
Every solution to the homogeneous

Let $A$ be an $n\times n$ matrix with rank $n-1$.  Suppose $u$ and $v$ in $\R^n$ are distinct solutions to the inhomogeneous system $Ax = b$.  Verify that every \RED{solution} to $Ax = b$ can be written as 
$\alpha u+(1-\alpha)v$ for some $\alpha\in \mathbb R$.

(Hint: idea is similar to Exercise $\ref{A.3.4.2}$.)

\begin{solution}
\soln 

\RED{Here is another possible solution.  I'm not sure that the needed theorems have already been stated. If correct the references to the needed theorems should be given.

Note that $u-v$ is a nonzero solution to the homogeneous system $Ax = 0$.  Since $\rank(A) = n-1$ every solution to the homogeneous system has the form $\alpha(u-v)$.  Therefore every solution to the inhomogeneous system has the form
\[
\alpha(u-v) + v = \alpha u + (1-\alpha) v
\]
as claimed.}


Since $A$ has rank $n-1$ and the system $Ax=b$ is consistent, the solution has one free variable, namely, the reduced echelon form of $A$ is 

\[
\Matrix{1 & 0 & 0 & \cdots &0 & a_1\\ 0 &1 &0   &\cdots &0&a_2\\ 0& 0 &1 &\cdots &0&a_3 \\ & \cdots & & \cdots& &\\ 0& 0 &0 &\cdots &1&a_{n-1}\\  0& 0 &0 &\cdots &0&0},
\]

where $a_i\in \mathbb R$ is an unknown number for $1\le i\le n-1$, so we can express the solutions of the homogeneous equation $Ax=0$ as 

\[
x=\Matrix{x_1 \\ x_2 \\ \vdots \\ x_{n-1} \\x_n}=\Matrix{-a_1x_n \\ -a_2x_n \\ \vdots \\-a_{n-1}x_n\\ x_n}
=x_n\Matrix{-a_1 \\ -a_2 \\ \vdots \\-a_{n-1}\\ 1},
\]

where $x_n$ is a free variable. 

On the other hand, by assumption, $u$ and $v$ are distinct solutions to $Ax=b$, so $u-v$ is a nonzero solution of the homogeneous equation $Ax=0$, then it should be a scalar multiple of $\Matrix{-a_1 \\ -a_2 \\ \vdots \\-a_{n-1}\\ 1}$. In other words, one can express any solution to the homogeneous equation $Ax=0$ as $x=\alpha(u-v)$, where $\alpha\in \mathbb R$. Thus any solution to the inhomogenous equation $Ax=b$ can be written as $\alpha(u-v)+v=\alpha u+(1-\alpha)v$.


\end{solution}
\end{exercise}


\begin{exercise} \label{YZ_3.4.6}
Let $L:\mathbb R^3\to \mathbb R^4$ be a mapping such that 
\[
Lx=\Matrixc{x_1-x_2 \\ x_1+x_2-x_3\\ -x_1+x_2+4x_3 \\ -3x_1-x_3}
\]
for all $x=\Matrix{x_1 \\ x_2 \\ x_3}.$
\begin{enumeratea}
\item Find matrix representative of $L$.
\item Verify that $x=\Matrix{-1/4\\-5/4\\-3/2}$ is a solution to the equation $Lx=\Matrix{1 \\ 0 \\ -7\\1 }$.
\item Find the full set of solutions of $Lx=\Matrix{1 \\ 0 \\ -7\\1 }$.
\end{enumeratea}


\begin{solution}

\ans
 \begin{enumeratea}
\item  The matrix representative of $L$ is 
\[
A=\Matrix{1 &-1& 0\\ 1&1&-1\\ -1&1&4 \\ -3&0&-1}.
\]
\item Direct verification.
\item $x=\Matrix{-1/4\\-5/4\\-3/2}$ is the only solution.
\end{enumeratea}


\soln \RED{NEED TO SHOW SOLUTIONS TO (a) (b) (c) }

We use the elementary row reduction to find the reduced echelon form of the matrix $A$:

\[
\Matrix{1 &-1& 0 \\ 1&1&-1 \\ -1&1&4  \\ -3&0&-1 } \to \Matrix{1 &-1& 0 \\ 0&2&-1 \\ 0&0&4 \\ 0&-3&-1} \to
\Matrix{ 1 &-1& 0\\ 0&2&-1 \\ 0&-3&-1 \\ 0&0&4  }
\]

\[
\to \Matrix{1 &-1& 0\\ 0&2&-1 \\ 0&0&-2 \\ 0&0&4 } \to \Matrix{1 &0& 0 \\ 0&1&0 \\ 0&0&1  \\ 0&0&0 }.
\]

Therefore, $A$ has rank 3, so $Ax=0$ only have zero solution, so by principle of superposition,  $x=\Matrix{-1/4\\-5/4\\-3/2}$ is the unique solution of the inhomogeneous equation $Ax=b$.
\end{solution}
\end{exercise}


\begin{exercise} \label{YZ_3.4.7}
Let $L:\mathbb R^3\to \mathbb R^2$ be a linear mapping such that 
\[
\Matrix{1\\-1\\0}=\Matrix{-3\\1},~L\Matrix{0\\1\\1 }=\Matrix{2\\-1},
\]
and 
\[
L\Matrix{ 0\\1\\-1}=\Matrix{0\\1}.
\]


\begin{enumeratea}
\item Determine the matrix representation of $L$. Namely, find the matrix $A$ such that $L_A=L$.

\item Verify that  $\Matrix{0\\2\\-2}$ is a solution to $Lx=\Matrix{0\\2}$
\item Find the full set of solutions of $Lx=\Matrix{0\\2}$.
\end{enumeratea}


\begin{solution}

\ans 
\begin{enumeratea}
\item  The matrix representation of $L$ is 
 \[
\Matrix{ -2&1&1\\1&0&-1}.
\]

\item Direct verification.
\item The general solution of the inhomogeneous equation is 
\[
\alpha \Matrix{1\\1\\1}+\Matrix{0\\2\\-2}, \alpha\in \mathbb R
\]

\end{enumeratea}



\soln 

\begin{enumeratea}
\item  $\Matrix{ 0\\0\\2 }=L\Matrix{ 0\\1\\1}-L\Matrix{0\\1\\-1}=\Matrix{2\\-2 }$, so 
\[
L\Matrix{0\\0\\1}=\Matrix{1\\-1}.
\]
Similarly, subtracting it from the second equation, we get 
\[
L\Matrix{0\\1\\0}=L\Matrix{ 0\\1\\1}-L\Matrix{0\\0\\1}=\Matrix{2\\-1}-\Matrix{1\\-1}=\Matrix{1\\0 }.
\]

Finally, adding it to the first equation, we have 
\[
L\Matrix{1\\0\\0}=L\Matrix{1\\-1\\0}+L\Matrix{0\\1\\0}=\Matrix{ -3\\1}+\Matrix{1\\0}=\Matrix{-2\\1}.
\]

Therefore the matrix representation $A$ of $L$ is 
\[
\Matrix{ -2&1&1\\1&0&-1}.
\]

\item \RED{Verify that 
\[
\Matrix{ -2&1&1\\1&0&-1} \Matrix{0\\2\\-2} =\Matrix{0\\2}.
\]
}

\item From the elementary row reduction of $A$, one can find the solution to $Ax=0$ is 
\[
\Matrix{1\\1\\1}.
\]
Therefore, the general solution of the inhomogeneous equation is 
\[
\alpha \Matrix{1\\1\\1}+\Matrix{0\\2\\-2}, \alpha\in \mathbb R
\]
\end{enumeratea}



\end{solution}
\end{exercise}







\end{document}
